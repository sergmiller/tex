\documentclass[12pt]{article}
% Эта строка — комментарий, она не будет показана в выходном файле
\usepackage{ucs}
\usepackage[utf8x]{inputenc} % Включаем поддержку UTF8
\usepackage[russian]{babel}  % Включаем пакет для поддержки русского языка
\usepackage{amsmath}
\usepackage{amssymb}
\usepackage{mathtools}

\title{Домашнее задание № 1}
\date{\today}
\author{Сергей Миллер 494}

\begin{document}
 	\maketitle
	\textbf{Задача 1.}
 	
  	\textbf{a)}
  	\begin{equation}
  	 	 b^{*}(a^{*}ab)^{*}a^{*} 	
   	\end{equation}

  	Рассмотрим произвольное слово $w$, несодержащее подслова $abb$, то есть после любого вхождения подслова $ab$ в $w$ следующий 			символ либо $a$, либо $\varepsilon$. Значит $w$ имеет вид 
  	\begin{equation}
  	 	 b \dots b a \dots aba \dots aba \dots aba \dots a 
   	\end{equation}
  	Также очевидно, что регулярное выражение (1) задает слова вида (2).
    	
  	\textbf{б)}
  	\begin{equation}
  		([(a^2 + b^2)^{*}(ab+ba)]^2)^{*}(a^2 + b^2)^{*}
  	\end{equation}
  	Анаогично рассмотрим произвольное слово $w$, подходящее под условие. Можно видеть, что это слово непосредственно разбивается на пары символов вида $aa$, $bb$, $ab$, $ba$ (в каждую пару входят 2$n$ и 2$n$ + 1 символы). Теперь можно заметить, что условие четности вхождения символов $a$ и $b$ в слово равносильно четности суммарного количества вхождений $ab$ и $ba$ в слово. А это значит, что слово можно разбить на четное число блоков, в которых сначала идет произвольное количество пар вида $aa$ и $bb$, а после один блок $ab$ или $ba$. А также после всех блоков опять может встретиться произвольное количество блоков $aa$ и $bb$. Нетрудно видеть, что данное регуярное выражение задает слова ровно такого вида.
  	
  	\textbf{в)}
  	\begin{equation}
  		([(a^2 + b^2)^{*}(ab+ba)]^2)^{*}(a^2 + b^2)^{*}[b + (ab+ba)(a^2 + b^2)^{*}a]
  	\end{equation}
  	Разобьем все подходящие слова $w$ на 2 вида: $w_1a$ и $w_1b$. Очевидно, что если $w = w_1b$, то $w_1$ описывается регулярным выражением предыдущей задачи.Для описания случая $w = w_1a$ составим регулярное выражение, анлогичное выражению из задачи 1б, только здесь будет нечетное число необходимых блоков(так как оно должно описывать слова с нечетным числом символов $a$ и $b$).

  \textbf{Задача 2.}
    Является.

    Рассмотрим два возможных случая: 
    \begin{eqnarray}
      \forall x \in \mathbb{N} \quad \exists p \in \mathbb{N}:  p \geq n, \quad p \in \mathbb{P}, \quad p+2 \in \mathbb{P}
    \end{eqnarray} 
    когда это верно и неверно. В первом случае в язык будут входить все слова вида $a$ \dots $a$. То есть язык будет описываться регулярным выражением $a^{*}$. Во втором случае длина слов языка ограничена $n_0 \in \mathbb{N}: \nexists p \geq n_0:  p \in \mathbb{P}, p+2 \in \mathbb{P}$
    А значит, язык будет регулярным, так как содержит конечное число слов.

  \textbf{Задача 3.}
    Рассмотрим произвольное слово $w \in \mathnormal{L}(1+e(fe)^{*}f)$. Если $w = \varepsilon$, то $w \in \mathnormal{L}((ef)^{*})$.
    Иначе  $w = w_{e_0} (w_{f_1} w_{e_1} \dots w_{f_n} w_{e_n}) w_{f_0}$, где $w_{f_i} \in \mathnormal{L}(f)$ и $w_{e_i} \in \mathnormal{L}(e)$. Видно, что это слово очевидным образом разбивается на части длиной 2, имеющие вид: $w_e w_f$. То есть $w \in \mathnormal{L}((ef)^{*})$. Обратное включение доказывается аналогично.
  
  \textbf{Задача 4.}
    Докажем оба включения. Пусть, сначала $w \in \mathnormal{L}(e)$. Проведем индукцию по длине слова $w$. База индукции: $w_0 \in \mathnormal{L}(e)$ имеющее минимальную длину, среди всех слов языка. Так как $e = ef + g$, то $w_0 \in \mathnormal{L}(ef)$ или $w_0 \in \mathnormal{L}(g)$. Но, тогда в первом случае $w_0 = w_e w_f$ где $w_e \in \mathnormal{L}(e), w_f \in \mathnormal{L}(f)$ , а так как $\varepsilon \notin \mathnormal{L}(f)$, то $|w_f| > 0$, что противоречит минимальности $w_0$. Значит $w_0 \in \mathnormal{L}(g) \subset \mathnormal{L}(gf^{*})$. База доказана. Теперь рассмотрим произвольное $w \in \mathnormal{L}(e)$. Аналогично базе либо $w \in \mathnormal{L}(g) \subset \mathnormal{L}(gf^{*})$, либо $w \in \mathnormal{L}(ef)$ и $w = w_e w_f$, а так как $|w_f| > 0$, то по предположению индукции $w_e \in \mathnormal{L}(gf^{*})$. Тогда очевидно, что и $w \in \mathnormal{L}(gf^{*})$.
    Обратное включение доказывается аналогичной индукцией по длине слова из рассматриваемого языка. Пусть  $w_0 \mathnormal{L}(gf*)$ имеет минимальную длину из всех слов этого языка. Видно, что $w_0 \mathnormal{L}(g)$ (так как иначе слово можно очевидным образом уменьшить). Откуда следует, что $w_0 \mathnormal{L}(e)$. Аналогично рассмотрим произвольное $w \mathnormal{L}(g)$, и либо $w \mathnormal{L}(g) \subset \mathnormal{L}(e)$, либо $w = w_{gf^{*}} w_f$, причем $w_{gf^{*}} \in \mathnormal{L}(e)$ по предположению индукции(так как  $|w_f| > 0$ и поэтому $|w_{gf^{*}}| < |w|$). Тогда очевидно, что $w \in \mathnormal{L}(e)$. Что и требовалось.

  \textbf{Задача 5.}

    \mathnormal{L} = 
    \begin{Bmatrix}
    w | \text{ в } w \text{ не встречается подстрока } aa, (w[-1] = b || (w = b \dots ba)
    \end{Bmatrix}

    Действительно все такие слова имеют два вида: $b \dots bab \dots bab \dots b$ и $b \dots ba$ . Поэтому во всех таких словах можно выделить несколько блоков: блок состоящий только из букв $b$, до первого вхождения $a$. Остальное слово разобьем на блоки вида: $ab \dots b$. Видно, что только последний блок может состоять ровно из одного символа $a$. Поэтому все слова такого вида описываются данным регулярным выражением. Обратное включение также очевидно: либо данное слово имеет вид $b^{*}a$, либо после каждой буквы $a$ идет непустой блок из $b$. То есть все слова, описываемые этим регулярным выражением, принадлежат языку \mathnormal{L}.
\end{document}